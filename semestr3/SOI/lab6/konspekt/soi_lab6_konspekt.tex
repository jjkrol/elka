\documentclass[12pt,a4paper]{article}

\usepackage[OT4]{polski}
\usepackage[utf8x]{inputenc}
\usepackage{ifthen}
\usepackage{anysize}
\usepackage{geometry}
\usepackage{fancyhdr}
\usepackage{verbatim}
\pagestyle{fancy}
\headheight 1cm
\marginsize{2cm}{2cm}{0.5cm}{2cm}
\lhead{SOI - laboratorium nr 6 - system plików} 
\rhead{Jakub Król, R3I2}
\fancyheadoffset{0cm}
% komendy
\newcounter{ind}
\newcommand*{\wylicz}[2]{#1_1,#1_2,\dots,#1_{#2}} 
\newcommand*{\ndots}[1]{
\setcounter{ind}{1}
\whiledo{\value{ind}<#1}{. \stepcounter{ind}}} 

\renewcommand{\arraystretch}{1.5}

\begin{document}

\section{Treść zadania}
Napisać w środowisku systemu Minix program w języku C realizujący podstawowe funkcje systemu plików, oraz skrypt demonstrujący wykorzystanie tego programu.

% ------ ZALOZENIA ------------------
\section{Założenia}
Napisany program ma realizować następujące funkcje:
\begin{itemize}
\item tworzenie wirtualnego dysku
\item kopiowanie pliku z dysku systemu \texttt{Minix} na dysk wirtualny
\item kopiowanie pliku z dysku wirtualnego na dysk systemu \texttt{Minix}
\item wyświetlanie katalogu dysku wirtualnego
\item usuwanie pliku z wirtualnego dysku
\item usuwanie wirtualnego dysku
\item wyświetlanie zestawienia z aktualną mapą zajętości dysku
\end{itemize}
Wirtualny dysk ma mieć strukturę jednopoziomowego katalogu

% ------- REALIZACJA ----------------
\section{Dysk wirtualny}
Organizacja dysku będzie analogiczna do uproszczonego systemu plików FAT:\\

\begin{tabular}{c|c|c|c}
Długość: & 1 blok & sizeof(int)*liczba bloków danych & pozostałe\\\hline
Zawartość & Sektor informacyjny & Tablica Alokacji & Dane
\end{tabular}\\

\subsection{Sektor informacyjny}
Sektor informacyjny zawierać będzie dane na temat:
\begin{itemize}
\item rozmiaru dysku wirtualnego
\item ilości wolnego miejsca na dysku
\item liczby plików na dysku
\item adresu pierwszego bloku, w którym zapisane są deskryptory plików
\end{itemize}

\subsection{Tablica alokacji}
Tablica alokacji zapewnia połączenie między blokami zajmowanymi przez jeden plik.
\subsection{Dane}
Część podzielona na bloki, zawierająca dane umieszczane na dysku wirtualnym.

\section{Realizacja}
Zostanie napisany moduł, zawierający funkcje odpowiedzialne za realizacje działań podanych w założeniach. Moduł ten będzie dołączony przez program odpowiedzialny za zarządzanie dyskiem wirtualnym i który, w zależności od parametrów wywołania, będzie używał funkcji z modułu do wykonywania żądanych operacji dyskowych.

\subsection{Struktury}
\paragraph{\texttt{fileDescriptor}}
Reprezentuje deksryptor pliku, przechowując dane na temat jego:
\begin{itemize}
\item nazwy
\item wielkości
\item bloku początkowego
\end{itemize}
\paragraph{\texttt{informationSector}}
Reprezentuje sektor informacyjny przechowując odpowiednie wartości
\paragraph{\texttt{fatElement}}
Reprezentuje element tablicy alokacji - zawiera:
\begin{itemize}
\item informację o numerze następnego bloku z nim powiązanego
\item 1, jeśli jest to ostatni element łańcuch
\item 0, jeśli blok jest pusty
\end{itemize}

\subsection{Funkcje}
\paragraph{\texttt{createVirtualDisk}}
Realizuje tworzenie dysku wirtualnego o zadanym rozmiarze
\paragraph{\texttt{deleteVirtualDisk}}
Usuwa wirtualny sysk
\paragraph{\texttt{copyFromVirtualToMinix}}
Kopiuje plik z dysku wirtualnego do dysku systemu \texttt{Minix}
\paragraph{\texttt{copyFromMinixToVirtual}}
Przeprowadza operację kopiowania pliku z dysku systemu \texttt{Minix} do dysku wirtualnego. Dba przy tym o kontrolę przekroczenia wielkości dysku wirtualnego oraz o zachowanie unikalności nazw plików.
\paragraph{\texttt{listFiles}}
Wyświetla wszystkie pliki znajdujące się w katalogu na dysku wirtualnym
\paragraph{\texttt{removeFile}}
Usuwa plik o zadanej nazwie 
\paragraph{\texttt{showMap}}
Pokazuje mapę zajętości
`
% --------------- TESTY ------------------
\section{Testy}
Poprawność wykonanego zadania zamierzam przetestować za pomocą skryptu powłoki sprawdzającego następujące aspekty programu:
\begin{itemize}
\item poprawność tworzenia i usuwania dysku wirtualnego
\item kopiowanie jednego lub więcej plików na dysk wirtualny
\item kopiowanie jedengo lub więcej plików z dysku wirtualnego
\item zabezpieczenie przed kopiowaniem pliku większego niż pojemność dysku
\item zabezpieczenie przed kopiowaniem kilku plików o łącznej wielkości większej niż pojemność dysku
\end{itemize}

\end{document}
