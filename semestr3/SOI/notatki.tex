\documentclass[12pt,a4paper]{article}

\setcounter{secnumdepth}{-1} 
\usepackage[OT4]{polski}
\usepackage[utf8x]{inputenc}
\usepackage{ifthen}
\usepackage{anysize}
\usepackage{geometry}
\usepackage{fancyhdr}
\pagestyle{fancy}
\headheight 1cm
%\footheight 1cm
\marginsize{2cm}{2cm}{0.5cm}{2cm}
\fancyheadoffset{0cm}

% komendy
\newcounter{ind}
\newcommand*{\ndots}[1]{
\setcounter{ind}{1}
\whiledo{\value{ind}<#1}{. \stepcounter{ind}}} 

\renewcommand{\arraystretch}{1.5}
\makeatletter
\renewcommand\paragraph{\@startsection{paragraph}{4}{\z@}%
  {-3.25ex\@plus -1ex \@minus -.2ex}%
  {1.5ex \@plus .2ex}%
  {\normalfont\normalsize\bfseries}}
\makeatother

% nieuniwersalne

\lhead{SOI - kolokwium}
\rhead{Jakub Król}
\begin{document}
	\section{Definicje podstawowe} % (fold)
	
	\paragraph{system operacyjny} % (fold)
	
	jest to zbiór programów i procedur spełniających dwie podstawowe funkcje:
	\begin{itemize}
		\item zarządzanie zasobami systemu komputerowego
		\item tworzenie maszyny wirtualnej
	\end{itemize}
	% paragraph system operacyjny (end)
	\paragraph{Zasób systemu} % (fold)
	
	każdy element sprzętowy lub programowy, który może być przydzielony danemu procesowi
	% paragraph Zasób systemu (end)
	\paragraph{Proces} % (fold)
	
	wykonujący się program wraz z jego środowiskiem obliczeniowym. Proces stanowi podstawowy obiekt dynamiczny w systemie operacyjnym
	% paragraph Proces (end)
	\paragraph{Semafor} % (fold)
	
	Zmienna inicjowana nieujemną wartością całkowitą, zliczająca sygnały \emph{wakeup}, zdefiniowana poprzez definicje niepodzielnych operacji P(s) i V(s):
		P(s):
		\begin{verbatim}
			while S leq 0 do
			;
			S:=-1
		\end{verbatim}
		V(s):
		\begin{verbatim}
S:=S+1
		\end{verbatim}
	% paragraph Semafor (end)
	\paragraph{Monitor} % (fold)
	
	Stanowi zbiór procedur, zmiennych i struktur danych, które są zgrupowane w specjalnym module. W kazdej chwili tylko jeden proces aktywny może przebywać w danym monitorze
	% paragraph Monitor (end)
	\paragraph{Sekcja krytyczna} % (fold)
	
		fragment programu, w którym występując instrukcje dostępu do zasobów dzielonych. Instrukcje tworzące sekcje krytyczne musza być poprzedzone i zakończone operacjami realizującymi wzajemne wykluczanie.
	% paragraph Sekcja krytyczna (end)
	% section Definicje podstawowe (end)

	\section{Procesy i Wątki} % (fold)
	
	\paragraph{Pojęcia} % (fold)
	
	\begin{itemize}
		\item współbieżność - nie musi razem
		\item równoległość - kiedys sie wykona fizycznie na raz
		\item rozproszoność - wiele srodowisk
	\end{itemize}
	% paragraph Pojęcia (end)
	\paragraph{Przejścia stanów} % (fold)
	
		\begin{itemize}
			\item uruchomiony w trybie użytkownika
			\item uruchomiony w trybie jądra
			\item gotowy, w pamięci
			\item wstrzymany, w pamięci
			\item gotowy, wymieciony
			\item wstrzymany, wymieciony
			\item wywłaszczony
			\item utworzony
			\item zombie
		\end{itemize}
	% paragraph Przejścia stanów (end)
	\paragraph{Procesy i wątki} % (fold)
	
		Grupa procesów:
		\begin{itemize}
			\item ochrona - zaleta
			\item komunikacja - wada
			\item zasoby - wada
			\item wydajność - wada (chyba że równolegle)
		\end{itemize}
		Wątki:
		\begin{itemize}
			\item ochrona - wada
			\item komunikacja - zaleta
			\item zasoby - zaleta
		\end{itemize}
	% paragraph Procesy i wątki (end)
	\paragraph{Wątki poziomu jądra i użytkownika} % (fold)
	
	\begin{itemize}
		\item z poziomy użytkownika - tablica wątków jest zawarta w obrębie danego procesu
		\item z poziomu jądra - tablica wątków jest zawarta w jądrze
	\end{itemize}
	% paragraph Wątki poziomu jądra i użytkownika (end)
	\paragraph{Wątki w systemie solaris} % (fold)
	
	\begin{itemize}
		\item Procesy - standardowo
		\item Wątki poz użytkownika - biblioteczne, niewidoczne dla jądra
		\item Procesy lekkie - odwzorowanie między wątkami poz użytkownika, a jądra (LWP:wątek jądra)
		\item wątki jądra
	\end{itemize}
	% paragraph Wątki w systemie solaris (end)
	\paragraph{Serwery usług} % (fold)
	
	\begin{itemize}
		\item Wielowątkowe - współbieżność, blokujące wywołania systemowe
		\item jednowątkowe - brak współbieżności, blokujące wywołania systemowe
		\item automaty skończone - współbieżnośc, nieblokujące wywołania systemowe, przerwania
	\end{itemize}
	% paragraph Serwery usług (end)
	\paragraph{Szeregowanie procesów} % (fold)
	
	Techniki:
	\begin{itemize}
		\item bez wywłaszczania
		\item z wywłaszczaniem
	\end{itemize}
	Dobre szeregowanie:
	\begin{itemize}
		\item sprawiedliwe
		\item zgodne z polityką
		\item wyrównane
	\end{itemize}
	Wsadowe:
	\begin{itemize}
		\item przepustowe
		\item czas w systemie
		\item wykorzystanie procesora
	\end{itemize}
	Interaktywne:
	\begin{itemize}
		\item czas odpowiedzi
		\item proporcjonalność
	\end{itemize}
	Czasu rzeczywistego:
	\begin{itemize}
		\item spełnianie wymagań
		\item przewidywalność 
		\item Dla zdarzeń okresowych: $\sum^m_{i=1} \frac{C_i}{P_i} \leq 1$
	\end{itemize}
	% paragraph Szeregowanie procesów (end)
	\section{Synchronizacja} % (fold)
	
	\paragraph{Wyścig} % (fold)
	
		Sytuacja, w której dwa lub więcej procesów wykonuje operację na zasobach dzielonych, a ostateczny wynik tej operacji jest zależny od momentu jej realizacji.
	% paragraph Wyścig (end)
	\paragraph{Warunki konieczne SK} % (fold)
	
	\begin{itemize}
		\item wewnątrz SK może przebywać tylko jeden proces
		\item jakikolwiek proces znajdujący się poza SK nie może zablokować innego procesu pragnącego wejść do SK
		\item każdy proces oczekujący na wejście do Sk powinien otrzymać prawo dostępu w rozsądnym czasie
	\end{itemize}
	% paragraph Warunki konieczne SK (end)
	\paragraph{Mechanizmy z aktywnym oczekiwaniem} % (fold)
	
	\begin{itemize}
		\item Blokowanie przerwań - źle, bo może zablokować i nie odblokować
		\item Zmienne blokujące - źle, bo wyścig
		\item Ścisłe następstwo - źle, bo zagłodzenie
		\item Algorytm petersona - połączenie powyższych, kolejne trzymane w tablicy
		\item TSL - poniżej
	\end{itemize}
	% paragraph Mechanizmy z aktywnym oczekiwaniem (end)
	\paragraph{Mechanizmy z zawieszaniem} % (fold)
	
	\begin{itemize}
		\item sleep/wakeup - sygnał może nie zostac przechwycony (wyścig) - blokada
		\item semafory
		\item monitory
		\item komunikaty
	\end{itemize}
	% paragraph Mechanizmy z zawieszaniem (end)
	\paragraph{TSL} % (fold)
	
	Zaimplementowana sprzętowo.
	\begin{itemize}
		\item Czyta zawartość słowa z pamięci do rejestru
		\item Zapamiętuje wartość rejestru w panięci
	\end{itemize}
	Operacje czytania i pisania niepodzielne
	% paragraph TSL (end)
	\paragraph{Pięciu filozofów} % (fold)
	
	
	% paragraph Pięciu filozofów (end)
	% section Synchronizacja (end)
	% section Procesy i Wątki (end)

\end{document}
