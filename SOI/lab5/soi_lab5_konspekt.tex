\documentclass[12pt,a4paper]{article}

\usepackage[OT4]{polski}
\usepackage[utf8x]{inputenc}
\usepackage{ifthen}
\usepackage{anysize}
\usepackage{geometry}
\usepackage{fancyhdr}
\usepackage{verbatim}
\pagestyle{fancy}
\headheight 1cm
\marginsize{2cm}{2cm}{0.5cm}{2cm}
\lhead{SOI - laboratorium nr 5 - zarządzanie pamięcią} 
\rhead{Jakub Król, R3I2}
\fancyheadoffset{0cm}
% komendy
\newcounter{ind}
\newcommand*{\wylicz}[2]{#1_1,#1_2,\dots,#1_{#2}} 
\newcommand*{\ndots}[1]{
\setcounter{ind}{1}
\whiledo{\value{ind}<#1}{. \stepcounter{ind}}} 

\renewcommand{\arraystretch}{1.5}
%\makeatletter
%\renewcommand\paragraph{\@startsection{paragraph}{4}{\z@}%
%  {-3.25ex\@plus -1ex \@minus -.2ex}%
%  {1.5ex \@plus .2ex}%
%  {\normalfont\normalsize\bfseries}}
%\makeatother
%

\begin{document}

\section{Treść zadania}
Zmienić domyślny algorytm przydziału pamięci w systemie minix na dający możliwość wyboru pomiędzy \emph{worst fit} a \emph{first fit}.
Zademonstrować i zinterpretować różnice w działaniu algorytmów.

% ------ ZALOZENIA ------------------
\section{Założenia}
Modyfikowane są tylko następujące pliki:
\begin{itemize}
\item \texttt{/usr/src/mm/proto.h}
\item \texttt{/usr/src/mm/alloc.c}
\item \texttt{/usr/src/mm/table.c}
\item \texttt{/usr/src/fs/table.c}
\item \texttt{/usr/include/minix/callnr.h}
\end{itemize}

% ------- REALIZACJA ----------------
\section{Realizacja}
\subsection{Funkcje systemowe}
Aby zaimplementować funkcje systemowe konieczne były następujące zmiany:
\begin{itemize}
	\item \texttt{mm/table.c} - oraz \texttt{fs/table.c} dodanie funkcji do tablicy
	\item \texttt{mm/proto.h} - dodanie prototypów funkcji w sekcji \texttt{alloc.c}
	\item \texttt{/usr/include/minix/callnr.h} - zwiększenie tablicy wywołań
	\item \texttt{mm/alloc.c} - dodanie kodu obu funkcji na końcu pliku
\end{itemize}
\subsubsection{hole\_map}
Funkcja zapisuje do bufora adres i wielkość kolejnych bloków pamięci, po czym kopiuje je do zadanego bufora za pomocą systemowej funkcji \texttt{sys\_copy}.
\begin{verbatim}
int do_hole_map(){

	phys_clicks buffer[1024];
	struct hole * hp;
	int i = 0;
	int counter; 

	int nbytes = mm_in.m1_i1;
	char * usr_buff = mm_in.m1_p1;
	int usr_pid = mm_in.m_source;

	hp = hole_head;
	if(hp == NIL_HOLE) return -1;

	if(nbytes%2==0) nbytes-=2; /* sprawdz czy jest miejsce dla 0 */
	counter = nbytes/2;

	do{
		buffer[i++]=hp->h_len; /* dlugosc */
		buffer[i++]=hp->h_base; /* adres */
		hp=hp->h_next; /* nastepny */
		counter--;
	}
	while(hp != NIL_HOLE && counter > 0);
	buffer[i]=0; /* dopisz zero */
	sys_copy(0,D,(phys_bytes) buffer, usr_pid, D, (phys_bytes) usr_buff, nbytes);
	return 0;

}
\end{verbatim}
\subsubsection{worst\_fit} 
Funkcja ustawia wartość zmiennej \texttt{worstFit}
\begin{verbatim}
int do_worst_fit(){
	if(mm_in.m1_i1 == 0 || mm_in.m1_i1 == 1)
		worstFit = mm_in.m1_i1;
	return mm_in.m1_i1;
}
\end{verbatim}
\subsection{Algorytm worst fit}
Implementacja polegała na modyfikacji funkcji \texttt{alloc\_mem} w pliku \texttt{alloc.c} i modyfikacji istniejącego algorytmu przydzielania pamięci w następujący sposób:
Dodania warunku, który w zależności od wartości zmiennej \texttt{worstFit} wybierał odpowiedni algorytm
Realizacji algorytmu \emph{worst fit}, który przeszukiwał wszystkie dostępne bloki pamięci i porównywał je z największym jaki dotąd znalazł. Po dotarciu do końca listy, zwracał największy znaleziony blok i zmniejszał jego długość i początkowy adres w liście (analogicznie dla istniejącego już algorytmu).
\begin{verbatim}
register struct hole * max_hole, *prev_max_hole;
if(hp!=NIL_HOLE){ /* jest miejsce? */
	max_hole = hp;
	prev_max_hole = NIL_HOLE;
}
else{
	return(NO_MEM); /* nie ma */
}

while(hp!=NIL_HOLE){
	if(hp->h_len > max_hole->h_len){
		prev_max_hole = max_hole;
		max_hole = hp;
		}
	hp = hp->h_next;
} /* while */
if(max_hole->h_len >= clicks){
	old_base = max_hole->h_base;
	max_hole->h_base += clicks;
	max_hole->h_len -= clicks;

	if(max_hole->h_len == 0 && prev_max_hole!=NIL_HOLE)
	del_slot(prev_max_hole, max_hole);
	return(old_base);
	}
\end{verbatim}
% --------------- TESTY ------------------
\section{Testy}
Poprawność wykonywanego zadania przetestowałem za pomocą programów przygotowanych w skrypcie laboratoryjnym: \texttt{t.c}, \texttt{w.c}, \texttt{x.c} oraz \texttt{skrypt\_testowy}.

% -------------------- WNIOSKI ----------------------------- %
\section{Wnioski}
Wyniki działania programu \texttt{skrypt\_testowy} są zgodne z przewidywaniami teoretycznymi. Po ich przeanalizowaniu widoczne stają się różnice w działaniu dwóch algorytmów.

W przypadku algorytmu \emph{first fit} kolejno uruchamiane procesy programu \texttt{x}, zajmują pamięć z pierwszego wystarczająco dużego, wolnego bloku, a po 10 sekundach, kiedy pierwszy uruchomiony program kończy działanie, po kolei zwalniają zajęte bloki pamięci, co widoczne jest na listingu.

Przy użyciu algorytmu \emph{worst fit}, widoczne jest, że każdy kolejny uruchomiony program \texttt{x}, zajmuje blok pamięci będący częścią ostatniego wolnego bloku, ktory jest największy.

\end{document}
