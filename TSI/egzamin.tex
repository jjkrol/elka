\documentclass[10pt,i,twocolumn,a4paper]{article}

%\usepackage{mathptmx}

\usepackage[OT4]{polski}
\usepackage[utf8x]{inputenc}
%\usepackage{fullpage}
%\usepackage{booktabs}
%\usepackage{graphicx} %grafiki
%\usepackage[all]{xy} % do prostych diagramów
%\usepackage{tikz}
%\usetikzlibrary{shapes,arrows}
%\usepackage{color}
\usepackage{amsmath} % matematyka
\usepackage{mathtools} % matematyka
\usepackage{amssymb} % symbole, np. triangleeq
%\usepackage{caption}
%\usepackage{subfig}
\usepackage{anysize} % marginesy
% header
\usepackage{fancyhdr}
\pagestyle{fancy}
\headheight 1cm
\marginsize{2cm}{2cm}{0.5cm}{2cm}
\lhead{TSI - egzamin}
\fancyheadoffset{0cm}


% z analizy
%---\usepackage{fullpage}
%\usepackage{amsfonts}
%\usepackage{amsmath}
%\usepackage{amssymb}
%\usepackage{txfonts}
%\usepackage{pxfonts}
%\usepackage{verbatim}
%\usepackage{multicol}
% --



% komendy
\renewcommand{\arraystretch}{1.5}
\newcommand*{\wylicz}[2]{#1_1,#1_2,\dots,#1_{#2}} 
\newcommand*{\wyliczn}[2]{\{#1_1,#1_2,\dots,#1_{#2}\}} 
\makeatletter
%\renewcommand\paragraph{\@startsection{paragraph}{4}{\z@}% %		{-3.25ex\@plus -1ex \@minus -.2ex}% %		{1.5ex \@plus .2ex}% %		{\normalfont\normalsize\bfseries}} %		\makeatother %


\begin{document}
\author{Jakub Król}
\date{Semestr zimowy 2011}
\section{Dyskretny} 
Wzór Byesa
\begin{align*}
	P(x_i|y_k) = \frac{P(y_k|x_i) P(x_i)}{P(y_k)}
\end{align*}
P-wo warunkowe
\begin{align*}
	P(A|B) = \frac{P(A.B)}{P(B)}
\end{align*}
Entropia
\begin{align*}
	H(x) = \sum P(x_i) \log_a{\frac{1}{P(x_i)}}
\end{align*}
Transinformacja
\begin{align*}
	I(\xi,\eta) & = H(\xi)-H(\xi|\eta) \\
				& =H(\eta)-H(\eta|\xi)\\
				& = H(\xi,\eta)-H(\eta)-H(\xi)
\end{align*}
Sprawność kodu
\begin{align*}
	\kappa = \frac{H(\xi)}{\overline{L}}
\end{align*}

\begin{align*}
	C = \log_2{M} - \left[\sum_{i=1}^M d_i\log_2{\frac{1}{d_i}} \right]
\end{align*}
Sprawność transmisji
\begin{align*}
	\kappa = \frac{I(\xi,\eta)}{C}
\end{align*}
Średnia długość słowa
\begin{align*}
	\overline{L} = \sum P(x_i)L(x_i)
\end{align*}
Prawdopodobieństwo wystąpienia jedynki
\begin{align*}
	P(1) = \frac{\sum L_i(1)P(x_i)}{\overline{L}}
\end{align*}
Szybkość transmisji
\begin{align*}
	v = \frac{H(\xi)}{\overline{L}t_0} = I\frac{H(\xi)v_0}{\overline{L}} = v_0H(\alpha) (\xi,\eta)f_s \overline{L}
\end{align*}

\section{Kod korekcyjny}
\begin{align*}
	m \le 2^k-k-1
\end{align*}

\section{Rozkład ciągły}
\begin{align*}
SNR = 10\lg{\frac{S}{n}}
\end{align*}
\begin{align*}
	y(t) = A(1+m\cos{2\pi f})\cos{2\pi f}
\end{align*}
Przepustowość łącza
\begin{align*}
	C_0 = \frac{1}{2}\log_2(1+\frac{S}{N})
\end{align*}
\begin{align*}
	C=B\log_2(1+\frac{S}{N})
\end{align*}
\begin{align*}
	C_{\infty} = \frac{S}{N_0}\log_2{e}
\end{align*}
Sygnał gaussowski
\begin{align*}
	f_\xi(x) = \frac{1}{\sqrt{2\pi \sigma^2}} \exp\left\{-\frac{(x-m)^2}{2\sigma^2}\right\}
\end{align*}
Moc sygnału
\begin{align*}
	S = \frac{1}{2} X_m^2 = \frac{1}{2} A^2 = \sigma^2
\end{align*}

\section{Filtry} 
\section{A/C} 
\begin{align*}
	SNR = 10 \lg(3\cdot2^{2b-1})
\end{align*}
\begin{align*}
	N = \frac{X_m^2}{3\cdot2^{2b}}
\end{align*}

Wartość kwantu
\begin{align*}
	q = \frac{X_m}{2^b-1}
\end{align*}


\section{Inne} 
\begin{align*}
	\kappa_{modulacji} = \frac{m^2}{2+m^2}
\end{align*}
\begin{align*}
	\frac{E_i}{N_0} \ge \ln{2}
\end{align*}

\end{document}

